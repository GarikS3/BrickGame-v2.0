\documentclass{article}
\usepackage{enumitem}
\usepackage{hyperref}
\usepackage[russian]{babel} 
\usepackage{fontenc}
\usepackage[T2A]{fontenc} 
\usepackage{times} 
\usepackage{helvet} 
\usepackage{courier} 

\usepackage{geometry}
\geometry{top=0cm, left=1cm, right=1cm, bottom=1cm, includeheadfoot}

\usepackage{titlesec}
\titlespacing*{\section}{0pt}{0pt}{0pt}
\titlespacing*{\subsection}{0pt}{0pt}{0pt}
\let\oldnewpage\newpage
\renewcommand{\newpage}{\oldnewpage}

\begin{document}

\title{Документация для проекта BrickGame v2.0 (Tetris + Snake)}
\date{}
\author{strangez}
\maketitle

\section{Структура проекта}

\subsection{Библиотека Tetris (\texttt{src/brick\_game/tetris})}

\begin{itemize}[label=--]
    \item Файлы с исходным кодом на языке C, реализующие логику игры Тетрис.
    \item Основные функции для взаимодействия с игровым полем, управлением фигурами и обработкой ввода пользователя.
\end{itemize}

\subsection{Библиотека Snake (\texttt{src/brick\_game/snake})}
\begin{itemize}[label=--]
    \item Файлы с исходным кодом на языке C++, реализующие логику игры Змейка.
    \item Основные функции для взаимодействия с игровым полем, управлением змейкой и обработкой ввода пользователя.
\end{itemize}

\subsection{Терминальный интерфейс (\texttt{src/gui/cli})}
\begin{itemize}[label=--]
    \item Файлы, отвечающие за визуализацию игр Тетрис и Змейка в терминале с использованием библиотеки ncurses.
    \item Отрисовка игрового поля, обработка ввода пользователя и отображение текущего состояния игры.
\end{itemize}

\subsection{Десктопный интерфейс (\texttt{src/gui/desktop})}
\begin{itemize}[label=--]
    \item Файлы, отвечающие за визуализацию игр с использованием библиотеки Qt.
    \item Отрисовка игрового поля, обработка ввода пользователя и отображение текущего состояния игры.
\end{itemize}

\section{Сборка проекта}
Проект использует систему сборки \texttt{make} с Makefile, включающим следующие цели:

\begin{itemize}
    \item \texttt{all}: Сборка проекта.
    \item \texttt{install}: Установка программы в систему.
    \item \texttt{uninstall}: Удаление программы из системы.
    \item \texttt{clean}: Очистка временных файлов и папок.
    \item \texttt{dvi}: Создание файла DVI.
    \item \texttt{dist}: Создание архива, содержащего необходимые файлы для сборки и использования программы.
    \item \texttt{gcov report}: Отчёт по покрытию кода.
    \item \texttt{clang check}: Проверка на необходимость форматирования кода.
    \item \texttt{clang format}: Форматирование кода.
    \item \texttt{valgrind}: Проверка на утечки памяти.
    \item \texttt{rebuild}: Удаление программы и новая установка.
\end{itemize}

\section{Инструкции по установке и запуску}

\begin{enumerate}
    \item \textbf{Установка зависимостей:}
        \begin{itemize}
            \item Установите компилятор gcc для сборки проекта.
            \item Установите библиотеку ncurses для консольного варианта.
            \item Установите библиотеку Qt для десктопного варианта.
        \end{itemize}
    \item \textbf{Сборка проекта:}
        \begin{itemize}
            \item Выполните \texttt{make all} для сборки проекта.
        \end{itemize}
    \item \textbf{Установка:}
        \begin{itemize}
            \item Выполните \texttt{make install} для установки программы в систему.
        \end{itemize}
    \item \textbf{Запуск:}
        \begin{itemize}
            \item Выполните \texttt{make run\_cli} для запуска программы в консольном варианте.
            \item Выполните \texttt{make run\_desktop} для запуска программы в десктопном варианте.
        \end{itemize}
\end{enumerate}

\section{Использование программы Тетрис}

\begin{enumerate}
    \item \textbf{Управление:}
        \begin{itemize}
            \item \texttt{keyleft}, \texttt{keyright} — перемещение фигуры влево и вправо.
            \item \texttt{keydown} — падение фигуры вниз.
            \item \texttt{SPACE} — поворот фигуры.
            \item \texttt{p} — постановка игры на паузу / снятие с паузы.
            \item \texttt{q} — выход из игры.
        \end{itemize}
    \item \textbf{Игра:}
        \begin{itemize}
            \item Вращение и перемещение фигур.
            \item Ускорение падения фигуры.
            \item Показ следующей фигуры.
            \item Уничтожение заполненных линий.
            \item Завершение игры при достижении верхней границы игрового поля.
        \end{itemize}
\end{enumerate}

\section{Использование программы Змейка}

\begin{enumerate}
    \item \textbf{Управление:}
        \begin{itemize}
            \item \texttt{keyleft}, \texttt{keyright}, \texttt{keyup}, \texttt{keydown} — перемещение змейки по игровому полю.
            \item \texttt{p} — постановка игры на паузу / снятие с паузы.
            \item \texttt{q} — выход из игры.
        \end{itemize}
    \item \textbf{Игра:}
        \begin{itemize}
            \item Управление направлением движения змейки.
            \item Генерация нового яблока на поле.
            \item Увеличение длины змейки при съедании яблока.
            \item Победа в игре в случае полного заполнения поля.
            \item Завершение игры при столкновении головы змейки с границами игрового поля или её собственным телом.
        \end{itemize}
\end{enumerate}

\section{Тестирование}

Проект включает в себя unit-тесты Тетриса с использованием библиотеки \texttt{check} и unit-тесты Змейки с использованием библиотеки \texttt{gtest}.

\end{document}
